% LaTeX Template für Abgaben an der Universität Stuttgart
% Autor: Sandro Speth
% Bei Fragen: Sandro.Speth@studi.informatik.uni-stuttgart.de
%-----------------------------------------------------------
% Modul beinhaltet Befehl fuer Aufgabennummerierung,
% sowie die Header Informationen.

% Überschreibt enumerate Befehl, sodass 1. Ebene Items mit
\renewcommand{\theenumi}{(\alph{enumi})}
\renewcommand{\theenumii}{(\roman{enumii})}
% (a), (b), etc. nummeriert werden.
\renewcommand{\labelenumi}{\text{\theenumi}}
\renewcommand{\labelenumii}{\text{\theenumii}}

% Counter für das Blatt und die Aufgabennummer.
% Ersetze die Nummer des Übungsblattes und die Nummer der Aufgabe
% den Anforderungen entsprechend.
% Gesetz werden die counter in der hauptdatei, damit siese hier nicht jedes mal verändert werden muss
% Beachte:
% \setcounter{countername}{number}: Legt den Wert des Counters fest
% \stepcounter{countername}: Erhöht den Wert des Counters um 1.
\newcounter{sheetnr}
\newcounter{exnum}

% Befehl für die Aufgabentitel
\newcommand{\exercise}[1]{\section*{Exercise \theexnum\stepcounter{exnum}: #1}} % Befehl für Aufgabentitel

% Formatierung der Kopfzeile
% \ohead: Setzt rechten Teil der Kopfzeile mit
% Namen und Matrikelnummern aller Bearbeiter
\ohead{Hui Zeng}
% \chead{} kann mittleren Kopfzeilen Teil sezten
% \ihead: Setzt linken Teil der Kopfzeile mit
% Modulnamen, Semester und Übungsblattnummer
\ihead{Business Analytics\\
Winter semester 2020/2021\\
Lecture Notes Summary }

\definecolor{comments}{rgb}{0.41,0.54,0.21}
\definecolor{code}{rgb}{0,0,0}
\definecolor{keyword}{rgb}{0.77,0.48,0.57}
\definecolor{number}{rgb}{0.153,0.5,0}
\definecolor{codeBack}{rgb}{0.85,0.85,0.85}
\definecolor{string}{rgb}{0.81,0.57,0.47}

\lstdefinestyle{stdCode}{
	backgroundcolor=\color{codeBack},   
	commentstyle=\color{comments},
	literate=*{0}{{\textcolor{number}{0}}}{1}%
         {1}{{\textcolor{number}{1}}}{1}%
         {2}{{\textcolor{number}{2}}}{1}%
         {3}{{\textcolor{number}{3}}}{1}%
         {4}{{\textcolor{number}{4}}}{1}%
         {5}{{\textcolor{number}{5}}}{1}%
         {6}{{\textcolor{number}{6}}}{1}%
         {7}{{\textcolor{number}{7}}}{1}%
         {8}{{\textcolor{number}{8}}}{1}%
         {9}{{\textcolor{number}{9}}}{1}%
         {.0}{{\textcolor{number}{.0}}}{1}% Following is to ensure that only periods
         {.1}{{\textcolor{number}{.1}}}{1}% followed by a digit are changed.
         {.2}{{\textcolor{number}{.2}}}{1}%
         {.3}{{\textcolor{number}{.3}}}{1}%
         {.4}{{\textcolor{number}{.4}}}{1}%
         {.5}{{\textcolor{number}{.5}}}{1}%
         {.6}{{\textcolor{number}{.6}}}{1}%
         {.7}{{\textcolor{number}{.7}}}{1}%
         {.8}{{\textcolor{number}{.8}}}{1}%
         {.9}{{\textcolor{number}{.9}}}{1}%
         {\ }{{ }}{1}% handle the space
         ,%
	keywordstyle=\color{keyword},
	numberstyle=\tiny\color{number},
	stringstyle=\color{string},
	basicstyle=\ttfamily\scriptsize,
	breakatwhitespace=false,         
	breaklines=true,                 
	captionpos=b,                    
	keepspaces=true,                 
	numbers=left,                    
	numbersep=5pt,                  
	showspaces=false,                
	showstringspaces=false,
	showtabs=false,                  
	tabsize=2
}

\lstset{
	style=stdCode, language=Java,
  	literate={ö}{{\"o}}1
           {ä}{{\"a}}1
           {ü}{{\"u}}1
}

\tikzset{triangle/.style = {regular polygon, regular polygon sides=3 },
node rotated/.style = {rotate=180},
border rotated/.style = {shape border rotate=180},
astTerminal/.style = {regular polygon, regular polygon sides=3, inner sep=2.5pt, shape border rotate=180},
astLabel/.style = {right=3pt,font=\footnotesize\itshape},
astValue/.style = {below=5pt},
astLine/.style = {edge from parent fork down}
}