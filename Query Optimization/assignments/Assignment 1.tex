\documentclass[12pt]{article}

\usepackage[english]{babel}
\usepackage[utf8x]{inputenc}
\usepackage[T1]{fontenc}
\usepackage{parskip}
\usepackage{lipsum}
\usepackage[a4paper, total={6in, 8in}]{geometry}
\usepackage{setspace}
\usepackage[superscript]{cite}
\usepackage{xcolor}
\usepackage{hyperref}
\usepackage{enumitem}
\usepackage{listings}
\usepackage{amsmath}

\definecolor{azzurro}{cmyk}{1,0.33,0,0.13}
\definecolor{arancione}{cmyk}{0,0.41,1,0}
\definecolor{verde}{cmyk}{0.44,0,0.38,0.31}
\definecolor{viola}{rgb}{0.58,0,0.82}
\definecolor{bianco}{rgb}{0.95, 0.95, 0.92}
% C style
\lstdefinestyle{CStyle}{
    backgroundcolor=\color{bianco},   
    commentstyle=\color{verde},
    keywordstyle=\color{viola},
    numberstyle=\tiny\color{arancione},
    stringstyle=\color{azzurro},
    basicstyle=\footnotesize,
    breakatwhitespace=false,         
    breaklines=true,                 
    captionpos=b,                    
    keepspaces=true,                 
    numbers=left,                    
    numbersep=5pt,                  
    showspaces=false,                
    showstringspaces=false,
    showtabs=false,                  
    tabsize=2,
    language=C
}
\lstset{
  basicstyle=\ttfamily,
  columns=fullflexible,
  frame=single,
  breaklines=true,
  postbreak=\mbox{\textcolor{red}{\(\hookrightarrow\)}\space},
}


% change margins within the geometry package and eventually the font size

\begin{document}
	
\begin{flushright}
	\today
\end{flushright}
{\Large \textbf{Assignment 1}}
	
{\large Query Optimization}
	
\textsc{Ilaria Battiston - 03723403} \\
\textsc{Mareva Zenelaj - 03736071}
	
\rule{\linewidth}{0.5pt}
	
\section{First exercise}
\subsection{SQL and relational calculus}

\subsubsection{SQL}
\begin{lstlisting}[language=SQL]
SELECT DISTINCT Studenten.MatrNr, Name
FROM Studenten, Hoeren, (
	SELECT VorlNr
	FROM Studenten, Hoeren
	WHERE Name = 'Schopenhauer'
	AND Studenten.MatrNr = Hoeren.MatrNr) Tmp
WHERE Studenten.MatrNr = Hoeren.MatrNr
AND Hoeren.VorlNr = Tmp.VorlNr
AND Name <> 'Schopenhauer'
	
SELECT DISTINCT PersNr, Name
FROM Professors, Vorlesungen, (
	SELECT VorlNr, COUNT(MatrNr)
	FROM Hoeren
	GROUP BY VorlNr
	HAVING COUNT(MatrNr) > 1) Tmp
WHERE Professors.PersNr = Vorlesungen.gelesenVon
AND Vorlesungen.VorlNr = Tmp.VorlNr
\end{lstlisting}

\newpage
\subsubsection{Tuple calculus}
Here we assume that the data structures are sets, therefore contain no duplicates.

$\{s\ : \ \{MatrNr,\ Name\}\ |\ s \in \text{Studenten} \land s.name \neq \text{'Schopenhauer'} \\
\land \exists h \in \text{Hoeren}(s.MatrNr = h.MatrNr) \\
\land \exists v \in (\{v\ |\ \text{Hoeren} \land \exists s \in (\text{Studenten} \land s.name = \text{'Schopenhauer'} \land s.MatrNr = v.MatrNr)\} \\ 
\land v.VorlNr = h.VorlNr)
\}$

$\{p\ : \ \{PersNr,\ Name\}\ |\ p \in \text{Professoren} \\
\land \exists v \in \text{Vorlesungen}(p.PersNr = v.gelesenVon) \\
\land \exists c1,\ c2 \in (\{c1,\ c2\ |\ \text{Hoeren} \\
\land c1.MatrNr \neq c2.MatrNr \land c1.VorlNr = c2.VorlNr \}) \\ 
\land c1.VorlNr = v.VorlNr\}$

\subsection{Relational algebra}
Here we assume that the data structures are sets, therefore contain no duplicates.

$\sigma_{S.Name \neq \text{'Schopenauer'}} \\
(S \times \sigma_{S.MatrNr = H.MatrNr} \\
(H \times \sigma_{H1.VorlNr = H.VorlNr} \\
(H1 \times \sigma_{H1.MatrNr = S1.MatrNr \land S1.Name = \text{'Schopenhauer'}}(S1))))$

$\sigma (P \times \sigma_{P.PersNr=V.gelesenVon} \\
(V \times (\sigma_{V.VorlNr=H1.VorlNr} \\ 
(H1 \times \sigma_{H1.MatrNr \neq H2.MatrNr \land H1.VorlNr = H2.VorlNr}(H2)))))$
	
\end{document}